\section{Results}

\begin{figure}
    \centering
    \includegraphics[width = 0.9\textwidth]{resources/images/generated/combined_heatmap.png}
    \caption{\small A heatmap with the \gls{nes} values for the geneset of interest (as rows) and the tested cohorts (as columns). The columns are clustered hierarchically. Opaque boxes are statistically significant (as reported by adjusted p-values of GSEA) to an alpha threshold of $0.20$, while semi-transparent ones are not. Coordinates with a dot are significant to the same alpha threshold when considering absolute fold-change values.}
    \label{fig:full_enrichment_heatmap}
\end{figure}

The enrichment matrix generated by the pipeline is visible in Figure \ref{fig:full_enrichment_heatmap}.
The matrix provides a summarized overview of the over- or under-expression of each transporte cohort.
As each leaf node can be considered meaningfully different from any other node, the matrix represents an eagle's-eye-view of the disregulations in the different cancer types.

In particular, it clear to notice how some tumor types closely cluster with various signatures:
\begin{itemize}
    \item Kidney, colorectal, stomach, Head and neck and brain cancer show generalized deregulation in many categories, and more specifically downregulation.
    \item Brain cancer shows downregulation or dysregulation in most categories.
    \item Pancreatic, esophageal, lung cancer and limphoma show little to no deregulation, with the exception of a strong downregulation of proton \glspl{ict} in lymphoma.
    \item Prostate, Testicular, Skin, Thyroid, Bladder, Female reproductive organs, adrenal gland and Breast cancers share a general upregulation of proton \glspl{ict}.
    \item Kidney cancer shows downregulation in a few categories, but shows significant deregulation when looking at absolute metric values across almost all genesets.
    \item The only cancer that shows a difference in expression of aquaporins is ovarian cancer.
\end{itemize}

Looking at the heatmap row-wise, we are able to appreciate other features:
\begin{itemize}
    \item With the exception of lymphoma and ovarian cancer, most cancer types show either no change or downregulation of pores.
    \item Transporters are generally deregulated.
    In most cancer types they are upregulated, and seem downregulated in only Kidney, Head and Neck and Brain cancer.
    \item Not many cancers (just three, colorectal, prostate and breast) show deregulation in \gls{abc} transportes, but more show absolute deregulation in the same category.
\end{itemize}