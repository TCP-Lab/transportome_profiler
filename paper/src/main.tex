\documentclass{article}
% Use `report` for longer reports

% Borders
    \usepackage{geometry}
    \geometry{
        a4paper,
        total={150mm,217mm},
        left=30mm,
        top=40mm,
    }
    
% Fonts
    \usepackage[no-math]{fontspec}
    \usepackage{unicode-math}
    \defaultfontfeatures{
        Scale=1.10
    }
    % Setting custom fontshttps://www.overleaf.com/project/643f9d6b8e8802d3523af0d2
    \setmainfont{Atkinson-Hyperlegible}[
    Path = resources/fonts/,
    Extension = .ttf,
    UprightFont = *-Regular-102,
    BoldFont = *-Bold-102,
    ItalicFont = *-Italic-102,
    BoldItalicFont = *-BoldItalic-102
    ]
    \setmonofont{FiraCode}[
    Path = resources/fonts/,
    Extension = .ttf,
    UprightFont = *-Regular,
    BoldFont = *-Bold
    ]
    
    \setmathfont{Latin Modern Math}[Scale=MatchLowercase]
    \setmathrm{Latin Modern Roman}[Scale=MatchLowercase]
    
    \setlength{\parskip}{0.75em}
    \renewcommand{\baselinestretch}{1.35}
    
    \usepackage[scaled=1.22]{helvet}

% Remove a bit of the space after headings
    \usepackage{titlesec}
    
    \titlespacing\section{0pt}{12pt plus 4pt minus 2pt}{0pt plus 2pt minus 2pt}
    \titlespacing\subsection{0pt}{12pt plus 4pt minus 2pt}{0pt plus 2pt minus 2pt}
    \titlespacing\subsubsection{0pt}{12pt plus 4pt minus 2pt}{0pt plus 2pt minus 2pt}
    
% Glossary
    \usepackage[acronym,toc]{glossaries}
    \loadglsentries{resources/glossary}
    \makeglossaries
    % Give a bit of colour to the glossary entries.
    \renewcommand{\glstextformat}[1]{{\color{glossary} #1}}
    
% Add dots to the TOC
    \usepackage{tocloft}
    \renewcommand{\cftsecleader}{\cftdotfill{\cftdotsep}}
    
% Support for hyperlinks
    \usepackage[hyphens]{url}
    \usepackage{hyperref}
    \hypersetup{
    colorlinks=true,
    linkcolor=exturlblue,
    urlcolor=urlblue,
    breaklinks=true,
    citecolor=citations,
    }
    \urlstyle{same}

% Bibliograpy
    \usepackage[
        backend=biber,
        style=nature
        ]{biblatex}
    \addbibresource{resources/bibliography.bib}
    
% Support for colors
    \usepackage{xcolor}
    \definecolor{listingback}{RGB}{225, 235, 252}
    \definecolor{commentgreen}{RGB}{24, 153, 76}
    \definecolor{numbergray}{RGB}{133, 133, 133}
    \definecolor{keywordorange}{RGB}{199, 86, 10}
    \definecolor{urlblue}{RGB}{35, 68, 107}
    \definecolor{exturlblue}{RGB}{0, 65, 145}
    \definecolor{glossary}{RGB}{0, 107, 18}
    \definecolor{mono}{RGB}{204, 71, 0}
    \definecolor{citations}{RGB}{123, 12, 179}
    
% Code listings
    \usepackage{listings}
    \lstdefinestyle{RStyle} {
        language=R,
        otherkeywords={!,!=,~,\$,*,\&,\%/\%,\%*\%,\%\%,<-,<<-,/,TRUE,FALSE},
        alsoother={.\$}
    }
    \lstdefinestyle{PyStyle} {
        language=Python
    }
    \lstdefinestyle{ShellStyle} {
        language=bash
    }
    % For inline monospace font: \lstinline{insert code here}
    \lstset{
        style=Rstyle,
        alsoletter=_,
        frame=tb,
        numbers=left,
        showspaces=false,
        basicstyle=\small,
        numberstyle=\tiny\color{numbergray},
        keywordstyle=\color{keywordorange},
        commentstyle=\color{gray},
        stringstyle=\color{commentgreen},
        backgroundcolor=\color{listingback}
    }
    % Pretty colours in mono font
    \newcommand{\mono}[1]{\texttt{\color{mono}#1}}
    
% Smaller title
    \makeatletter         
    \renewcommand\maketitle{
        {\raggedright % Note the extra {
        {\huge \bfseries \sffamily \@title }\\[2ex]
        {\small \@author}
        }} % Note the extra }
    \makeatother
    
% Images
    \usepackage{graphicx}
    \graphicspath{ {./resources/images/} }
    \usepackage{float} % To use hard 'H' 
    % Style of images
    % Set a max width to all images
    \usepackage[export]{adjustbox}% http://ctan.org/pkg/adjustbox
    \let\oldincludegraphics\includegraphics
    \renewcommand{\includegraphics}[2][]{%
        \IfFileExists{#2}
        {\oldincludegraphics[#1,max width=\linewidth]{#2}}
        {\oldincludegraphics[#1,max width=\linewidth]{example-image}}
        
    }
    % Support for subfigures
    \usepackage{caption}
    \usepackage{subcaption}

% Better lists
    \usepackage{enumitem}
    \setlist{nosep}
    
% Highlight boxes
    \usepackage{awesomebox}

% Multiple authors
    \usepackage{authblk}

% Making appendices
\newcommand{\printappendices}{
    % Reset the section/figure/table counter 
    \setcounter{figure}{0}
    \setcounter{section}{0}   
    \setcounter{table}{0}   
    % Set the figure/table counter to aleph
    \renewcommand\thefigure{\Alph{figure}}
    \renewcommand\thesection{Appendix \Alph{section}}
    \renewcommand\thetable{\Alph{table}}
    % Set the chapters to use letters
    \appendix
    % Set the section numberings
    \newpage
    % Add a large (clickable) title
    \addcontentsline{toc}{part}{Appendices}
    \part*{Appendices}
    \section{Example Appendix}

Hey, I'm an appendix.

}

% Add TODO notes
    \newcommand{\todo}[1]{\textbf{\color{red}#1}}

% Footers and headers
    \usepackage{fancyhdr}
    \pagestyle{fancy}
    \fancyhead[]{}
    \fancyhead[LO]{\color{red}Draft generated from source code on \today}
    \renewcommand{\headrulewidth}{0pt} % no line in header area
    \fancyfoot{} % clear all footer fields
    \fancyfoot[RO]{\thepage}           % page number in "outer" position of footer line
    \fancyfoot[LO]{\color{red}Draft generated from source code on \today}

% Better abstracts
\usepackage{abstract}
\renewcommand{\abstractname}{\normalfont \large \textbf{Abstract}}

% Better Tables
    \usepackage{tabularx}

% My proposal for a title: The transcriptional landscape of the Transportome in Cancer
    \title{\normalfont Profiling the Expression of Transportome Genes in cancer: A systematic approach}
    \author[*]{Luca Visentin\footnote{L.V., luca.visentin@unito.it}}
    \author[**]{Giorgia Scarpellino\footnote{G.S., giorgia.scarpellino@unipv.it}}
    \author[*]{Luca Munaron\footnote{L.M., luca.munaron@unito.it}}
    \author[*]{Federico Alessandro Ruffinatti\footnote{Corresponding Author: F.A.R., federicoalessandro.ruffinatti@unito.it}}

    \affil[*]{Department of Life Sciences and Systems Biology, University of Turin, Turin, Italy}
    \affil[**]{Department of Biology and Biotechnology “L. Spallanzani”, University of Pavia, Pavia, Italy}
    
\begin{document}

\maketitle

%Suppress the table of contents
%\tableofcontents

\begin{abstract}
    \large
    The transportome, the -omic layer encompassing all ion channels and transportes (ICTs), is crucial for cell physiology.
    It is therefore reasonable to hypothesize a role of the transportome in disease, and in particular in cancer.
    Here, we present the Membrane Transport Protein database (MTP-db), a database collecting information on ICTs, and a pipeline to take expression data and the MTP-db as input and produce a broad overview over the disregulated transportome in cancer.
    The MTP-DB may prove useful for the study of the transportome in general, and the pipeline may be used to study the transportome in other diseases.
    Both tools are open source and can be found on Github at \href{https://github.com/TCP-Lab/MTP-db}{TCP-Lab/mtp-db} and \href{https://github.com/TCP-Lab/transportome_profiler}{TCP-Lab/transportome\_profiler}, under permissive licenses.
    We detect that the transportome is disregulated in cancer, and that dysregulation patterns are shared between different cancer types.
    It is still unclear how these disregulation patterns are linked with cancer physiology.
\end{abstract}

\newpage

\section{Introduction}

The movement of molecules between the intracellular and extracellular environments is essential for cell survival.
This movement is due, in part, by transmembrane proteins of various nature, which we refer to with the term \textit{transmembrane transporters}.
The -omic layer that comprises the transmembrane transporters is referred to as the \textit{transportome}.

The coordinated action of these proteins regulates a large number of physiological functions, such as membrane potential, nutrient absorption, waste product removal, cellular signalling, regulation of intracellular and extracellular pH, and more.

The key molecule that performs the action of transportation is of course the protein.
The presence of the specific transport protein itself is, however, not sufficient for the achievement of the overall function.
The presence of accessory proteins or other complex-forming proteins is often a requirement for the correct formation of the so-called \textit{\gls{fu}}, which is then able to perform the function proper.

Cataloguing functional units is complex, especially due to the current challenges in the proteomics field.
Once sufficient technological advancements are made, however, it is probable that functional units will be defined with high accuracy.
Currently, however, a generally accepted approximation to the functional unit is the gene transcript, assuming that the presence and quantity of the transcript is directly proportional to the \glspl{fu}.

There are many exceptions, but it is easy to distinguish between:

\begin{itemize}
    \item Pores: \gls{fu} that create water-filled pores that allow the facilitated passage of molecules through the membrane.
    These may be additionally subdivided into \textit{channels} proper and \textit{aquaporins}, pores that allow only the passage of water.
    \item Transporters: proteins or protein complexes that, embedded in the membrane, allow passage of molecules thanks to changes in protein conformation.
    These may be furthermore divided into those that require the hydrolysis of ATP to function and those that do not as \textit{atp-driven transporters} and \textit{solute carriers}, respectively.
    \begin{itemize}
        \item A common distinction in atp-driven transporters is given by \textit{\gls{abc} transporters} and \textit{pumps}, where \gls{abc} transporters feature the conserved \gls{abc} subunit, while pumps do not.
    \end{itemize}
\end{itemize}

A commonly carried out essay is the measurement of gene expression in healthy and diseased samples, followed by differential expression analysis and gene list enrichment with ontologies such as the \gls{go} with tests such as Fisher's exact test.
This process is, of course, dependent on the curation and structure of the \gls{go}.

A similar, but "reversed" approach is to generate gene lists of interest and then test them against the data to see if the list is differentially expressed or not, for example through the \gls{gsea} method.
Assuming that these gene lists meaningfully group together genes by their functional role, the main advantage of such an approach is that through the systematic generation of all possible lists (given some descriptors), one could in theory profile all aspects of the studied function.

Cancer cells differ fundamentally in respect with their healthy counterparts, especially in their relationship with the extracellular environment.
Dramatic metabolic shifts are also seen in cancer.
Both of these aspects probably involve a shift in transportome \glspl{fu}, and are potentially reflected in the expression levels of the same genes.

We therefore aimed at profiling the expression levels of transportome genes in the context of cancer.
To do this, we collected information on these genes (such as their complete list, which molecule(s) they transport, their gating mechanism, their functional class, etc.) and used it to systematically arrange them into meaningful \glspl{tgl}.
After sorting all the protein-coding genes found in cancer cells based on their differential expression with respect to healthy cells, we run a pre-ranked \gls{gsea} on these ordered lists to obtain enrichment scores for every \gls{tgl}.
We therefore obtained the "deregulation status" of most functional facets of the transportome in 19 different cancer tissue types.

We provide an open-source, documented and reproducible Python package, Daedalus (\href{https://github.com/CMA-Lab/MTP-DB}{github.com/CMA-Lab/MTP-DB}) that retrieves transportome-related data from various databases and compiles it in a local \mono{.sqlite} database, and we provide pre-compiled database files as periodic releases.

We additionally provide a \mono{make}-driven and docker-containerized pipeline, dubbed "transportome profiler", (\href{https://github.com/CMA-Lab/transportome_profiler}{github.com/CMA-Lab/transportome\_profiler}) that takes gene expression data and the aforementioned database to generate gene lists, sort genes based on their differential expression and run \gls{gsea}.
This pipeline was designed with modularity and reproducibility in mind, so that it would be easily adaptable on other datasets and databases.


\section{Materials and Methods}

In this section we broadly cover the implementation of the pipeline, with particular regard to the generation of the \glspl{tgs} and the related \gls{gsea}.

\subsection{The Membrane Transport Protein Database}

% I want to introduce the MTP with the answer to "why didn't you use the GO?"
Even if the \gls{go} knowledgebase \cite{ashburnerGeneOntologyTool2000,thegeneontologyconsortiumGeneOntologyKnowledgebase2023} represents an invaluable source of meaningful and readily available gene sets, we decided not to rely solely on the manually curated and automatically generated gene lists of the \gls{go}.
Instead, we preferred to explore all the possible \glspl{tgs} that can originate by the transporter-related features of interest in a more unbiased and systematic way.

To improve the ease of use and portability, we compiled all the data we needed to generate the \glspl{tgs} in a SQLite database and named it \gls{mtpdb}.
To populate the \gls{mtpdb}, we sourced information from a variety of online databases.
In particular, we extracted data from % This is like this by design!
the \gls{hgnc} database \cite{sealGenenamesOrgHGNC2023},
the \gls{iuphar} ``Guide to Pharmacology'' database \cite{hardingIUPHARBPSGuide2022},
Ensembl \cite{cunninghamEnsembl20222022} through the use of BioMart \cite{smedleyBioMartBiologicalQueries2009},
the \gls{tcdb} \cite{saierTransporterClassificationDatabase2021},
the \gls{slc} Tables website \cite{hedigerABCsMembraneTransporters2013},
the \gls{cosmic} \cite{tateCOSMICCatalogueSomatic2019} database,
and the \gls{go} \cite{ashburnerGeneOntologyTool2000, thegeneontologyconsortiumGeneOntologyKnowledgebase2023}.

\begin{table}
\begin{tabularx}{\textwidth}{|X|X|X|}
        \hline
    \textbf{Database / Repository}  & \textbf{Access method}
        & \textbf{Retrieved Data} \\
        \hline
        \hline
    HGNC Database                   & FTP repository, direct download URLs
        & Gene group classification, Gene Names and Symbols (indirectly) \\
        \hline
    IUPHAR / Guide to Pharmacology  & Direct download URLs
        & Gene group classification, Ion Channel selectivity and conductance
        metrics, protein ligand information, ligand-to-protein interactions \\
        \hline
    Ensembl (Biomart)               & BioMart XML requests
        & Gene group classification, Gene Names, Gene Identifiers (Ensembl Genes,
        Transcripts and Protein IDs, RefSeq Gene and Protein IDs, pdb IDs) \\
        \hline
    TCDB                            & Direct download URLs
        & Gene group classification \\
        \hline
    SLC Tables (BioParadigm)        & Webpage Scraping
        & Solute carriers selectivity information \\
        \hline
    COSMIC                          & Direct download URL
        & Mutational Information \\
        \hline
    Gene Onthology                  & BioMart XML requests
        & Ion Channel selectivity information \\
        \hline
\end{tabularx}
\caption{List of accessed databases, methods of access, and the use of the
    downloaded data in the \acrlong{mtpdb}.}
\label{tab:dataSources}
\end{table}

The specific data downloaded from each database and the methods used to retrieve them can be seen in Table \ref{tab:dataSources}.
A Python package nicknamed \textit{Daedalus} was created to download, parse, and save this information, recreating the \gls{mtpdb} \textit{on the fly} at every execution.
We opted for this generation method instead of a manually updated, static database file for a few reasons:
a dynamically regenerated database will always be up-to-date in respect of the upstream database changes,
it does not require particular hosting capabilities to be accessed even in the far future (as the users can simply download and run the source code to obtain a copy),
and it provides a clear, transparent, and traceable way to see how the source data is obtained, parsed, and stored (by examining the source code).

It was interesting to notice how information from the above databases was, and still is at the time of writing, flawed in several minor ways.
For instance, the GABA\textsubscript{A} receptors in the brain are ionotropic receptors notoriously permeable to chloride ions, but the \gls{hgnc} database only classifies them as \textit{ligand gated ion channels} and not as \textit{chloride channels}.
Additionally, the \gls{iuphar} does not contain any permeability information for any of the \textit{GABR\textasteriskcentered{}} genes (conductance and permeability being properties of the receptor \textit{protein complex} rather than the single subunit gene).
This makes it impossible for Daedalus to classify them as permeable to chloride ions.
It is also known that GABA\textsubscript{A} receptors exhibit a certain permeability to HCO\textsubscript{3}\textsuperscript{-}, generally in between $0.2$ and $0.4$ that of the chloride ions \cite{goetzGABAAReceptors2007}, but not even this feature is reflected in the above databases.
A similar argument applies to nicotinic acetylcholine receptors (nAChRs) and related \textit{CHRN\textasteriskcentered{}} subunits.
Moreover, when classifying channels by \textit{gating mechanism}, HGNC makes no distinction between \textit{gating} proper (in the senses of activating stimulus) and \textit{modulation}.
As a result, some voltage-modulated channels, such as the TRP superfamily of cation channels are pooled with pure voltage-activated channels such as K\textsubscript{v}, Na\textsubscript{v}, and Ca\textsubscript{v}, which may not be desirable from a physiologist's point of view.
A final example is how some ion channels (including many beta subunits of potassium voltage-gated and calcium-activated channels,as well as the mitochondrial calcium uniporter family of proteins) were missing from the ion channel lists provided by the \gls{iuphar} and the \gls{hgnc}.
All of these defects were manually addressed in the \gls{mtpdb} through post-build hooks automatically applied by Daedalus during each build process.
Notably, even such ``terminal patches'' can be easily inspected in detail by the reader in order to get an idea of the degree of manual curation within the \gls{mtpdb}.
If needed, post-build hooks can be turned off to obtain a version of the database free of manual annotations.

To support a more varied search strategy, especially when querying data for \glspl{slc}, we implemented a sort of internal thesaurus, adding synonyms or more general terms to the database besides the already-present information on the carried molecule in all relevant tables.
For example, we inserted the more general values of ``cation'' and ``anion'' next to each particular ion species;
we added more general synonyms such as ``carbohydrate'' for mannose and glucose;
``amino acid'' for all basic amino acids; and so on.
The thesaurus is also consulted to correct any parsing errors made by Daedalus, and manually addresses some imperfections in the source data (e.g., different codes for the same amino acid are all converted to the three-letter code).
The thesaurus is contained in a single, human-readable comma-separated-values file, available in the remote repository.

The \gls{mtpdb} will be an effective way to stimulate further research in the Transportome by providing a centralized and human- and machine-friendly place to obtain a lot of relevant information on Transportome genes.
The pre-compiled \gls{mtpdb} and the Daedalus package are open-source and available for free on GitHub at \href{https://github.com/TCP-Lab/MTP-DB/}{www.github.com/TCP-Lab/MTP-DB}, and we encourage users to inspect and propose improvements.

At every development milestone, a published version of the database (in the form of the generated SQLite file) will be released on GitHub, for reproducibility purposes.
The latest release can be downloaded directly from \href{https://github.com/TCP-Lab/MTP-DB/releases/latest/download/MTPDB.sqlite.gz}{this URL} in \mono{gzipped} format.
Additionally, note that the data blob downloaded and then used by Daedalus to generate the database is locally saved as a Python Pickle file, and is the only file that the package needs to regenerate the database.
This can be useful to exactly reproduce the database in the future even after the remote databases have updated.
The database is versioned using the CalVer specification, with representation \mono{Major.YY.0W[\_MINOR][-TAG]}.

At the time of writing, the latest version of the database is \mono{1.23.26-beta}, featuring a total of $932$ ICT gene entries ($356$ ion channel, $14$ aquaporins, $47$ ABC transporters, $83$ pumps, and $432$ \glspl{slc}). 

The \gls{mtpdb} is, however, still partially incomplete.
We manually curated a list of $151$ calcium-permeable ion channels and found that, although most of them were correctly annotated as such, $68$ ($45\%$) were not.
Of these, $30$ were annotated as ``cation'' permeable, but without the more specific term for calcium.
In that case, much of the inconsistency stemmed from the different interpretation of the term ``calcium channel'' given by the list curators.
\gls{hgnc}, for example, reserves this term for only those ion channels \textit{selectively} permeable to Ca\textsuperscript{2+}.
On the contrary, our list included all calcium-permeable channels, even those non-selective for that ion species.
Nevertheless, some well-known calcium-selective ion channels (including \textit{PIEZO1}, \textit{PIEZO2}, \textit{P2RX7}, and some others) were not correctly annotated in any of the sources consulted by Daedalus.
%Even now, even with post-build hooks, there are still 45 calcium channels not included: the gap junctions are still missing, pannexins, nicotinic cholinergic receptors, and ionotropic serotonin receptors... a next good issue.
We addressed these missing annotations, but we cannot however be sure that the same phenomenon does not occur for other ion channels, or even more generally  in other transporters.
For this reason we set up the GitHub repository with a collaborative approach
in mind, following Open Science practices.
This would allow researchers that notice an issue with the annotations in the database to participate in the creation of a more useful resource.

A docker image per every release, representing the environment in which the database was rebuilt in, is available on Docker Hub at \href{https://hub.docker.com/r/cmalabscience/mtpdb}{cmalabscience/mtpdb}.
Please note, however, that Daedalus does not directly depend on any environment variables.

\begin{table}
\begin{tabularx}{\textwidth}{|l|X|}
        \hline
    \textbf{Table name}
        & \textbf{Description} \\
        \hline
        \hline
    gene\_ids
        & Genome-wide gene identifiers from Ensembl (``ENSGs'') and their
        versions. \\
        \hline
    transcript\_ids
        & Transcript identifiers from Ensembl (``ENSTs''), and their version,
        associated with gene identifiers. \\
        \hline
    mrna\_refseq
        & RefSeq transcript identifiers and their ENST counterparts. \\
        \hline
    protein\_ids
        & Protein identifiers from Ensembl (``ENSPs'') associated with their
        respective ENSTs, PDB accession codes and RefSeq protein IDs. \\
        \hline
    gene\_names
        & ENSGs associated with their respective HUGO gene IDs, symbols and
        names. \\
        \hline
    iuphar\_targets
        & Every gene considered as a ``target'' by the IUPHAR, with relevant
        names, IDs and family information. \\
        \hline
    iuphar\_ligands
        & Every ligand considered by the IUPHAR, as well as its type, name, ENSG
        (if relevant), pubchem SIDs and CIDs, and information on drug status. \\
        \hline
    iuphar\_interaction
        & Every interaction between a IUPHAR-recognized ligand and human protein
        target, with relevant information. \\
        \hline
    tcdb\_ids
        & The entirety of the TCDB linked with the respecitve ENSPs, split by
        type, subtype, family, subfamily and superfamily. \\
        \hline
    tcdb\_types
        & Names of the types in the TCDB. \\
        \hline
    tcdb\_families
        & Names of the families in the TCDB. \\
        \hline
    cosmic\_genes
        & Information regarding the presence of the gene in the COSMIC database,
        as well as its hallmark status. \\
        \hline
    channels
        & All genes catalogued as ``channels'' by IUPHAR or the HGNC, with
        information on their gating mechanism, carried solute and relative and
        absolute conductance. Information from GO regarding permeability information is also included. \\
        \hline
    aquaporins
        & All aquaporin genes as well as their expression tissue. \\
        \hline
    solute\_carriers
        & All genes catalogued as as ``solute carriers'' by IUPHAR or the HGNC,
        as well as their carried solute, rate of transport, stoichiometry, port
        type, direction, and more. \\
        \hline
    pumps
        & All active transporters (excluding ABC transporters), correlated with their carried solute, rate,
        transport direction, net transported charge, and more. \\
        \hline
    ABC\_transporters
        & All ABC transporter genes, correlated with their carried solute, rate,
        direction, and more. \\
        \hline
\end{tabularx}
\caption{Brief description of the data contained in all the tables present in
    the \gls{mtpdb}.}
\label{tab:databaseSchema}
\end{table}

An overview of the database schema, with information on the type of data in every table, can be seen in Table \ref{tab:databaseSchema}.
The actual database schema file is present in the GitHub repository at
\href{https://github.com/TCP-Lab/MTP-DB/blob/main/src/daedalus/local_data/schema.sql}{this URL}.

\subsection{Generation of Transporter Gene Lists}

We created a Python script to query the database file produced by Daedalus and generate the \glspl{tgs}.
We structured the \glspl{tgs} as a tree in which each node represents a separate \gls{tgs}.
All transportome genes are present in the root node, while subsets of these gene sets are considered in every child node.

% -- Un po' avulso in questo punto... lo sposterei
%It is hard to differentiate between transcript of which proteins are embedded in the plasma membrane or in intracellular membranes based on the information in the DB.
%For this reason, the cellular localization of the transporters is currently ignored for the generation of \glspl{tgs}.

To divide the genes into meaningful \glspl{tgs}, we seeded the algorithm to always produce a manually defined tree---or ``core'' tree---reflecting the classification schema already shown in Figure \ref{fig:BasicTree}.
This subdivision purposefully follows also the internal schema of the database, since most nodes represent a table or a combination of whole tables.
The algorithm then retrieves all relevant information to each node of the core tree and finds all the possible sub-classes based on the available data, finally appending each sub-class to the tree as a child node.
This approach is then repeated for each added child, recursively creating smaller and smaller child gene sets.

In practice, for each core node, the database is queried to obtain the relevant columns of data to the starting node, possibly joining multiple tables.
For instance, for the ``ATP-Driven'' core node, the algorithm will consider data from both the \mono{ABC\_transporters} and the \mono{pumps} tables.
Once the data is retrieved from the database, for each available column the algorithm generates gene groups of unique genes with the same shared column value, appending them to the parent node.
The algorithm then considers a subset of the original data, based on the column value used to generate the child gene set, and attempts to create more by considering all other columns in turn.
This approach essentially enumerates all possible gene sets that could be
meaningfully derived from the annotated data.

This iterative process generates a very large number of gene sets (with version 1.23.25-beta of the database, this number was $\sim 6000$).
For this reason, we implemented a pruning procedure on the resulting tree, which prunes away leaf nodes that are highly similar with other nodes in the tree.

We measured node similarity by the Sørensen–Dice coefficient $\kappa$ as defined below:
\begin{equation}
    \kappa\left(A,B\right) = 2\ \frac{\left|A^g \cap B^g\right|}{\left|A^g\right|+\left|B^g\right|}\ ,
    \label{eq:node_similarity_1}
\end{equation}
where $A$ and $B$ are nodes in the tree $T$, associated to the sets of genes $A^g$ and $B^g$, respectively.
$\kappa$ is a value ranging from $0$ to $1$, where $1$ means perfectly congruent nodes, and $0$ meaning completely different nodes.
Given a tree $T$, we calculated the similarity index for each node $A$ as
\begin{equation}
    K_T\left(A\right) = \max_{B \in T,\ B \neq A}\kappa\left(A,B\right)
    \label{eq:node_similarity_2}
\end{equation}
If $K_T\left(A\right)$ is greater than a user-defined threshold value $\eta$ (generally close to $1$), the node $A$ is deemed redundant and pruned.
After all leaves are considered, if the algorithm pruned at least one leaf, it is executed again.
This process returns a tree in which all leaf nodes are sufficiently different from each other.
On the contrary, internal nodes are always preserved, even in the presence of high similarity scores between them, since they cannot be eliminated without the removal of meaningful leaf nodes.

% I know it's a bit on the nose - the 'acute' - but I love it.
The acute reader might notice that, as $\kappa$ is a symmetric measure, this process removes leaves in an order-dependent way.
Indeed, if $\kappa\left(A,B\right) > \eta$, the pruning of $A$ or $B$ depends on which is considered first.
This was taken into account, and the leaf nodes are sorted based on their depth (the number of edges to follow to return to the root node) before the pruning cycle begins.
This allows the usage of two pruning strategies: bottom-up or top-down.
In the top-down approach, higher leaves (farther from the root) are pruned first, as the nodes are sorted in descending order.
The opposite happens in the bottom-up approach, where leaves closer to the root are pruned first, in favor of distant leaves.

% Here below I slightly modified your beautiful example just to enable the comparison of the A->B->C sequence with its reverse... which is more closely akin to our situation :-)
Do note that although $\kappa$ is symmetric, similarity among sets is not a transitive relation (except for $\kappa = 1$), so the two pruning methods do not necessarily generate trees with an identical number of nodes.
This is not surprising, as three nodes $A$, $B$, and $C$, where $\kappa(A,B) > \eta$, $\kappa(A,C) > \eta$, but $\kappa(B,C) < \eta$ will be pruned differently if they are considered in one order (e.g., $A \rightarrow B \rightarrow C$, resulting in $\left\{B,C\right\}$) or the opposite (e.g., $C \rightarrow B \rightarrow A$, resulting in $\left\{A\right\}$).

For efficiency purposes, the algorithm does not consider data columns that contain too many missing values (deeming them not meaningful to split on, with the default being $> 50\%$ emptiness), does not generate gene sets that are smaller than a threshold (by default $10$ genes) and does not consider for further subsetting gene sets smaller than a threshold (by default $40$).

The specifications for the core tree, the SQL calls used to retrieve the data for each node in the core tree, the threshold for $\kappa$, the pruning direction, and other settings are all easily configurable as command-line arguments.
This makes this approach easily adaptable to future versions of the \gls{mtpdb} and even other kinds of SQLite and more generally SQL-based databases provided by other researchers.
For our run, we used the default parameter values, while $\kappa$ was set to $0.9$.

Once generated, the tree is translated to a set of files in folders on disk.
The folders are named with the names of the nodes, and the \mono{all.txt} files in them contain the Ensembl Gene IDs of the genes in the specific \gls{tgs}.
This structure was chosen to easily allow the reconstruction of the tree by any other programs, simply by following the tree-like file system paths.

\subsection{GSEA analysis}

Raw expression data (in the form of un-normalized RNA-Seq counts) and related metadata to accomplish this analysis were downloaded from the University of California Santa Cruz's Xena platform \cite{UCSCXena}, selecting the \mono{TCGA TARGET GTEx} study (although TARGET samples were omitted from the analysis).
This data have the benefit of being processed by the same analysis pipeline, eliminating technical noise coming from the pipeline itself.
% Here cite Xena TOIL publication:
%   John Vivian, et al.
%   Toil enables reproducible, open source, big biomedical data analyses.
%   Nat Biotechnol. 2017 Apr 11;35(4):314-316. doi: 10.1038/nbt.3772.
% If we run it barebones add version info. If we dockerize it, omit it.
% TODO: Do this - but not now for the preprint. Later for the full publication.

To run \gls{gsea}, we needed to pair \gls{tcga} tumor expression data with their \gls{gtex} ``healthy'' counterparts and then perform \gls{dea} on each cohort separately with the DESeq2 R package.
This proved to be not so straightforward.
The Xena platform provides a fused version of the \gls{gtex} official metadata file plus the \gls{tcga} metadata for ease of use.
However, it is unclear how to meaningfully pair such large databases, as methodology, possible batch effects due to sample handling, sample preprocessing (such as microdissection), and the intrinsic non-healthy nature of \gls{gtex} samples can cause confounding in the results.

To pair the healthy and tumor data together, we therefore followed the macroscopic grouping provided by the metadata files in the Xena platform.
To subset the very large expression matrix file based on the metadata, we implemented a Python package called \textit{metasplit}, which makes use of the low level Rust-compiled \mono{xsv} package to speed up the computation \cite{gallantBurntSushiXsv2023}.
The specific calls used to subset the expression matrix and therefore match the tumor and healthy samples are available in the GitHub repository at \href{https://github.com/TCP-Lab/transportome_profiler/blob/main/src/run_dea/tcga_gtex_queries.json}{this url}. %Should we explicitly show them as a table??
As an overview, we compared tumor samples with their healthy counterparts from the same tissue or organ of origin.

% Is the following correct? -- We might want to talk about this later, or earlier.
This comparison is purposefully generic.
As we lack specific information on the actual location of the biopsied tissue, its specific microdissection status, and potentially other experimental biases or variables at play, we deemed to be safer not to compare different tumor types of the same organ by themselves, but combine them into a more general category.

% We might want to discuss a bit about what the comparisons tell us, like we discussed internally.

The fitted DESeq2 model had only the tumor status variable and the intercept.
This resulted in an independent Wald-statistic metric for each gene in each comparison.
We used the Wald-statistic values as weights to run a weighted, pre-ranked GSEA analysis with the fGSEA R package \cite{korotkevichFastGeneSet2021} using the previously generated \glspl{tgs} of interest.
We ran the analysis with both actual values for the statistic and absolute ones, in order to also detect possible symmetric dysregulations.
Thus we obtained one significance table for each tumor-healthy cohort, with enrichment information for each input gene set.
Finally, we used this data to generate a global enrichment matrix with gene sets on the rows, the cohorts on the columns, and the \gls{nes} as values.

%% We will not include this for now -- but we will later!
%In order to see if independently-published dataset from single studies would
%compare with the results from the \gls{tcga}-\gls{gtex} analysis, we also
%gathered \todo{ADD NUMBER} independent datasets on \gls{pdac} samples from
%\gls{geo}, and subjected them to the same pipeline by comparing them to
%\gls{gtex} samples labelled as ``Pancreas''. The full table of dataset names and
%the studies they were sourced is available in Table \todo{ADD TABLE FROM
%GIORGIA}.
%%


\section{Results}

\begin{figure}
    \centering
    \includegraphics{resources/images/generated/full_heatmap.png}
    \caption{A heatmap with the \gls{nes} values for the geneset of interest (as rows) and the tested cohorts (as columns). The columns are sorted following hierarchical clustering. Opaque boxes are statistically significant (as reported by adjusted p-values of GSEA) to an alpha threshold of $0.20$, while semi-transparent ones are not.}
    \label{fig:full_enrichment_heatmap}
\end{figure}

Enrichment results for all the genesets across all the cohorts taken into account is visible in Figure \ref{fig:full_enrichment_heatmap}.


Test!

\section{Discussion}

\section{Conclusions}

\section{Other Statements}

\subsection{Author Contributions}

\subsection{Funding}

\subsection{Data and Code Availability}
Combined TCGA and GTEX data and metadata is available from the Xena platform at
\href{URLhttps://toil-xena-hub.s3.us-east-1.amazonaws.com/download/TcgaTargetGtex_gene_expected_count.gz }{this url}. The 

\subsection{Conflicts of Interest}
The Authors declare no conflicts of interest.



\newpage
\Urlmuskip=0mu plus 1mu\relax % This allows for better linebreaks in URLs
\printbibliography

\printglossaries

% Uncomment this to add appendices
%\printappendices

\end{document}
